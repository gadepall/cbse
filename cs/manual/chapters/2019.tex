\renewcommand{\theequation}{\theenumi}
\renewcommand{\thefigure}{\theenumi}
\begin{enumerate}[label=\thesection.\arabic*.,ref=\thesection.\theenumi]
\numberwithin{equation}{enumi}
\numberwithin{figure}{enumi}
\numberwithin{table}{enumi}

\item  Reduce the following Boolean Expression to its simplest form using K-Map :
\begin{align}
F(P,Q,R,S) = \sum (0,1,2,3,5,6,7,10,14,15) 
\end{align}
\solution  The logic expression is obtained from Fig. \ref{fig:2019_6d}
as
\begin{equation}
 F = P^{\prime}Q + RS^{\prime}+ P^{\prime}S + QR
\end{equation}
 %
\begin{figure}[!ht]
\centering
\resizebox{\columnwidth}{!}
{
\begin{karnaugh-map}[4][4][1][][]
    \maxterms{4,8,9,11,12,13}
    \minterms{0,1,2,3,5,6,7,10,14,15}
    \implicant{0}{2}
    \implicant{1}{7}
    \implicant{2}{10}
    \implicant{7}{14}
    % note: position\implicant{1}{7} for start of \draw is (0, Y) where Y is
    % the Y size(number of cells high) in this case Y=2
    \draw[color=black, ultra thin] (0, 4) --
    node [pos=0.7, above right, anchor=south west] {$RS$} % YOU CAN CHANGE NAME OF VAR HERE, THE $X$ IS USED FOR ITALICS
    node [pos=0.7, below left, anchor=north east] {$PQ$} % SAME FOR THIS
    ++(135:1);
        
    \end{karnaugh-map}
}
\caption{}
\label{fig:2019_6d}
\end{figure}
\end{enumerate}
%
%
